\begin{comment}
\chapter{Budgets}
\label{ch:budg}

\section{Contingency Management}
\label{sec:cont_mana}

The following section shows the contingency allowance for the mass and power budget, which are key elements in the design process. 

\noindent \autoref{tab:cont_mass} shows the mass budget contingency allowance, where the total mass was divided into four categories. It can be noted that the structure includes higher contingency factors, which is a consequence of it being highly dependent on most systems. Batteries, engines and electronics are expected to be off-the-shelf components and therefore more predictable. \\

\begin{table}[H]
\centering
\caption{Mass budget contingency allowance}
\label{tab:cont_mass}
\begin{tabular}{lllll}
\bottomrule
Design maturity         & Structure & Batteries & Propulsion & Electronics \\
\midrule
Preliminary design      & 10 \%     & 5 \%      & 5 \%    & 5 \%       \\ \bottomrule
\end{tabular}
\end{table}

In \autoref{tab:cont_powe} the same procedure was repeated for the power budget. In this case it is the propulsion system which deserves special attention, because it has to be electric and enable VTOL operations. This is one of the main challenges of the project and, compared to standard systems, more likely to exceed initial estimations on the power consumption.

\begin{table}[H]
\centering
\caption{Power budget contingency allowance}
\label{tab:cont_powe}
\begin{tabular}{lllp{2.0cm}l}
\toprule
Design maturity         & Propulsion & Communication & Actuators & Instrumentation \\\midrule
Preliminary design      & 10 \%      & 5 \% & 5 \%& 5 \%      \\ \bottomrule
\end{tabular}
\end{table}
\end{comment}