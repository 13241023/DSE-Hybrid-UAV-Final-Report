\chapter{Conclusion \& Recommendations}
\setlength{\parindent}{15pt}
\label{ch:conc_reco}

%purpose of report, also quickly summarise previous (20 conc)
The aim of this report is to document the continuation of the design process and to justify the choices that were made in the preliminary design phase. The concept used, the Winged Quadcopter, was determined in the midterm report \cite{midterm}. It was considered best after a trade-off between five options, as brought forward in the baseline report \cite{baseline}. The goal of the final design is to adhere to the mission need statement:
\begin{quote}
\itshape `Carry out both remotely controlled and supervised autonomous monitoring and transport missions, comprising vertical take-off and landing, and sustained high-velocity horizontal flight.'
\end{quote}

The design was carried out by five specialist teams, resulting in a concurrent engineering structure where design parameters are constantly updated after each design stage. These teams are aerodynamics, structure, command and data handling, power \& propulsion and stability \& control. Each had their corresponding tasks, and were the responsible team for these.

To start off the designing, a trade-off between five preliminary concepts was performed in order to determine what type of Hybrid UAV layout is optimal for the given requirements. The most important requirements are top speed of 200 km/h, endurance of at least an hour, a payload carrying capacity of 10 kg and a capability to take-off and land vertically. These requirements have been set by the client. However, it is determined that it is unfeasible to meet all these requirements simultaneously while adhering to the EASA mass limitation of 25 kg. To solve this, modular payloads are used, replacing payload weight with batteries if necessary. To prevent exceeding limits on mass, cost, and power, budgets have been implemented for each subsystem. To have an allowable margin of freedom, contingencies have been added to each budget. Subsystems were not allowed to exceed these budgets, if this was the case, changes had to be made to the design or budget. 


With the preparation done, the five departments worked concurrently on the various subsystems to design the most optimal outcome. Using a master set document with all updated variables, each department worked with the most up-to-date values. The final outcome resulted in a design that even exceeds some of the requirements. The aircraft is able to fly 40 km with a full payload, while being able to fly up to a maximum of 600 km by replacing payload weight with extra batteries. The requirement of having an endurance for an hour has been met, but not with the full payload capacity. The structure of the aircraft will be built with lightweight materials such as glass fibre and carbon fibre to ensure the mass of the aircraft will not exceed 25 kg. To conclude, the design meets the requirements using a modular payload, while staying within the budgets assigned to the subsystems.


\paragraph{Recommendations} With the preliminary design, the overall layout and approximate parameters are known.  For future design stages, the team has some suggestions on things that can be improved. For example, the fuselage can be made shorter, decreasing drag. Besides that, in the current design a modular payload is used to optimise performance per mission type. In the future, modularity can also be applied to the wings, props and engines. Also, more extensive CFD analysis should be done to enhance the aerodynamic design. Stability should be investigated more thoroughly, including dynamic stability properties. Finally, when more Hybrid UAVs are sold, some components that are currently off-the-shelf, such as the flight control module, can be replaced by in-house designed ones.

