\chapter{SAMPLE}
\pagenumbering{arabic}

This template has been developed for the [AE3200] Design Synthesis Exercise. \texttt{THIS TEMPLATE CANNOT BE USED WITHOUT EXPRESSED PERMISSION FROM:} \c{S}an K{\i}lk{\i}\c{s} and Munyung Kim

\section{Tables \& Figures}
An example \autoref{tab:exampletable} and an example \autoref{fig:flame} can be found in this section. When you label tables or figures, make sure to use `tab:name' or `fig:name'. For inserting 2+ figures in a row, look at the formatting of Figure \autoref{fig:sbs}. Using the \texttt{autocite} command negates the need for manually typing `Table' or `Figure'. The syntax is as follows, note that the `tab' in `tab:exampletable' is not necessary for \texttt{autocite} and is purely for organizational reasons.

\begin{verbatim}
    \autoref{tab:exampletable}
\end{verbatim}

The Tables below use the package \texttt{tabularx} which adjusts column spacing automatically to fit the table within the margins of the page. The syntax is as follows where 'L' is for Left Aligned, 'C' for Centered, and 'R' is for Right Aligned. If you wish to make a long table with \textit{longtable}, an example can be seen after the \textit{tabularx} example. You can use a normal tabular environment unless you really want an equal column spacing.

\begin{verbatim}
    \begin{tabularx}{\textwidth}{L C C C}
\end{verbatim}

\begin{verbatim}
	\begin{longtable}[htb]{llccc}
	\caption{Task division of the mid-term report}\label{tab:taskdivi}\\
\end{verbatim}

For longtable, do NOT use \textit{\centering}, as longtable by default puts the table in center. In order to keep up the same appearance for all tables use the commands \texttt{toprule}, \texttt{midrule}, \texttt{bottomrule}, and \texttt{hdashline} to create the horizontal lines. NO VERTICAL LINES ARE ALLOWED!

\begin{table}[H]
	\centering
	\caption{Example Table}
	\label{tab:exampletable}
	\begin{tabularx}{\textwidth}{L C C C} %'L' for Left Aligned, 'C' for Centered, 'R' for Right Aligned
	    \toprule\
		\textbf{Component}				& \textbf{Mass [$\text{kg}$]}	&\textbf{Location [$\text{m}$]} & \textbf{Location [\% MAC]}   \\ \toprule
		Wing 							& 425.4 						& 5.74 							& 40.00\\\hdashline
		Main Landing Gear 				& 243.1 						& 5.82 							& 45.00 \\\hdashline
		Fuel System 					& 80.74 						& 5.91 							& 50.00 \\\hdashline
		Flight Control System 			& 48.61 						& 6.08 							& 60.00	\\\hdashline
		Hydraulics 						& 4.660 						& 6.08 						    & 60.00 \\\hdashline
		\textbf{Wing Group} 			& \textbf{802.5} 				& \textbf{5.80} 				& \textbf{43.85}\\ \midrule
		Fuselage 						& 265.2 						& 5.74 	                        & 40.00 \\\hdashline
		Engine 							& 409.4 						& 1.64 							& - \\\hdashline
		Avionics 						& 490.9 						& 4.39 							& - \\\hdashline
		H. Tail 						& 42.93 						& 13.2 						    & - \\\hdashline
		V. Tail 						& 66.43 						& 12.6 						    & - \\\hdashline
		Nose Gear						& 54.58 						& 2.50 							& - \\\hdashline
		Electrical 						& 217.4 						& 6.16 							& 67.12 \\\hdashline
		AC \& Anti-Ice 					& 215.7 						& 6.16 							& 67.12 \\\hdashline
		Furnishings 					& 241.5 						& 6.16 							& 67.12 \\\hdashline
		\textbf{Fuselage Group} 		& \textbf{2004} 				& \textbf{5.01} 				& \textbf{-2.32} \\ \midrule
		\textbf{OEW C.G.} 				& \textbf{2806} 				& \textbf{5.24} 				& \textbf{10.88} \\ \bottomrule
	\end{tabularx}
\end{table}

Here's how you make a long table that breaks automatically over pages (just copy this if you want to make one)
\begin{verbatim}
    \begin{longtable}[]{l l p{.3\linewidth}} %'l' is left aligned, but for pieces with text use p{size} to define column width
        \caption{Long tables are awesome\label{tab:haha_fun}}\\ %putting label below gives an error and you need the enter\\
        %now define the layout of the table
        \toprule
        This is the text that only will go at the beginning of your table (so just once)
        \\ \midrule
        \endfirsthead
        %%%%
        \caption[]{(continued)}
        \toprule
        This is something that goes on top of each part on every page the table spans
        \\ \midrule
        \endhead
        %%%%
        \midrule \endfoot
        \bottomrule \endlastfoot
        Now & this & is\\
        the & place & where\\
        the & table & goes \\
    \end{longtable}
\end{verbatim}

\begin{table}[H]
	\centering
	\caption{Example Table II}
	\label{tab:exampletableII}
    \begin{tabularx}{\textwidth}{C C C C C C C C} %'L' for Left Aligned, 'C' for Centered, 'R' for Right Aligned
    \toprule\
    {$m$} & {$\Re\{\underline{\mathfrak{X}}(m)\}$} & {$-\Im\{\underline{\mathfrak{X}}(m)\}$} & {$\mathfrak{X}(m)$} & {$\frac{\mathfrak{X}(m)}{23}$} & {$A_m$} & {$\varphi(m)\ /\ ^{\circ}$} & {$\varphi_m\ /\ ^{\circ}$} \\ \toprule
    1  & 16.128 & +8.872 & 16.128 & 1.402 & 1.373 & -146.6 & -137.6 \\ \hdashline
    2  & 3.442  & -2.509 & 3.442  & 0.299 & 0.343 & 133.2  & 152.4  \\ \hdashline
    3  & 1.826  & -0.363 & 1.826  & 0.159 & 0.119 & 168.5  & -161.1 \\ \hdashline
    4  & 0.993  & -0.429 & 0.993  & 0.086 & 0.08  & 25.6   & 90     \\ \midrule
    5  & 1.29   & +0.099 & 1.29   & 0.112 & 0.097 & -175.6 & -114.7 \\ \hdashline
    6  & 0.483  & -0.183 & 0.483  & 0.042 & 0.063 & 22.3   & 122.5  \\ \hdashline
    7  & 0.766  & -0.475 & 0.766  & 0.067 & 0.039 & 141.6  & -122   \\ \hdashline
    8  & 0.624  & +0.365 & 0.624  & 0.054 & 0.04  & -35.7  & 90     \\ \midrule
    9  & 0.641  & -0.466 & 0.641  & 0.056 & 0.045 & 133.3  & -106.3 \\ \hdashline
    10 & 0.45   & +0.421 & 0.45   & 0.039 & 0.034 & -69.4  & 110.9  \\ \hdashline
    11 & 0.598  & -0.597 & 0.598  & 0.052 & 0.025 & 92.3   & -109.3 \\ \bottomrule
    \end{tabularx}
\end{table}

\begin{figure}[H]
    \centering
    \includegraphics[width=0.3\textwidth]{SAMPLE/Figures/flame.jpg}
    \caption{TU Delft Logo Flame}
    \label{fig:flame}
\end{figure}

\begin{figure}[H]
\centering
\begin{subfigure}[b]{0.5\textwidth}
  \centering
  \includegraphics[width=.85\textwidth]{SAMPLE/Figures/flame.jpg}
  \subcaption{TU Delft Logo Flame}
  \label{fig:flame1}
\end{subfigure}%
\begin{subfigure}[b]{0.5\textwidth}
  \centering
  \includegraphics[width=.85\textwidth]{SAMPLE/Figures/flame.jpg}
  \subcaption{TU Delft Logo Flame}
  \label{fig:flame2}
\end{subfigure}
\caption{Two Figures Side-by-Side} %Main Caption
\label{fig:sbs}
\end{figure}


\section{References \& Citations}
The \texttt{biblatex} package is used for references with the default `numeric' style for in-text citations and references. The references sorting style is set to `none' meaning that the references are sorted by the order in which they appear in text. A sample file \texttt{samplerefs.bib} is included to help when dealing with different types of publications. For using footnotes, use the footnote code at a location you wish to insert a footnote \footnote{This is an example}.

\begin{verbatim}
    \cite{citationtag}
\end{verbatim}

\begin{verbatim}
    \cite{\footnote{}}
\end{verbatim}



\section{Equations \& Nomenclature}
Starting with variables, you need to use a nomenclature code when you introduce a variable for the FIRST time, such that the variable is listed on the list of symbols. An example is given in Equation \autoref{eq:exampleeq}.

\begin{equation}
\label{eq:exampleeq}
    L = \frac{1}{2}\rho V^2 S \cdot C_{L}
\end{equation}

\nomenclature[A]{ABCD}{ABCD}
\nomenclature[B]{$C_L$}{Lift Coefficient \nomunit{-}}
\nomenclature[B, 01]{$V$}{Velocity \nomunit{\si{kg.m^{-1}}}}
\nomenclature[B, 02]{$S$}{Wing Area \nomunit{\si{m^{2}}}}
\nomenclature[G]{$\rho$}{Density of Air \nomunit{\si{kg.m^{-3}}}}

The the list of symbols for the above equation were generated with the code below:

\begin{verbatim}
    \nomenclature[A]{ABCD}{ABCD}
    \nomenclature[B]{$C_L$}{Lift Coefficient \nomunit{-}}
    \nomenclature[B, 01]{$V$}{Velocity \nomunit{\si{kg.m^{-1}}}}
    \nomenclature[B, 02]{$S$}{Wing Area \nomunit{\si{m^{2}}}}
    \nomenclature[G]{$\rho$}{Density of Air \nomunit{\si{kg.m^{-3}}}}
\end{verbatim}






