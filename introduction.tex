\chapter{Introduction}
\setlength{\parindent}{15pt}
\label{ch:intr}

%include MNS & POS in this chapter
In recent years, significant advances have been made in both the capabilities of automated control systems and the miniaturisation of technological components, both of which have been attributed to the exponential growth in the Unmanned Aerial Vehicle market over the last decade. Tasks that traditionally required constant human supervision, control and even physical effort are steadily being replace by UAVs, which signals the rise of UAVs being used in commercial applications.

Despite significant advances being made in the UAV market, what has proven to be challenging is combining the high velocity horizontal flight capabilities of fixed wing aircraft with the vertical take-off and landing and hovering capabilities typical of rotor aircraft. Hybrid UAVs aim to seamlessly combine the two and this challenge is what necessitates the research, calculations, and design documented in this report.

The project of which this report is the outcome, aims to document and present an efficient Hybrid UAV design which can be controlled remotely within visual line of sight (VLOS), with the capability of beyond visual line of sight (BVLOS) control in the future. Derived from the aforementioned goal is the project objective statement (POS) which summarises the goal of the project and the steps required to achieve it, and is characterised as follows:

%\nomenclature[A]{UAV}{Unmanned Aerial Vehicle} commented as defined in summary
%\nomenclature[A]{VTOL}{Vertical Take-off and Landing} commented as defined in summary
%\nomenclature[A]{VLOS}{Visual Line Of Sight}
%\nomenclature[A]{BVLOS}{Beyond Visual Line Of Sight}
\nomenclature[A]{POS}{Project Objective Statement}
\nomenclature[A]{MNS}{Mission Need Statement}

\begin{quote}
	\begin{itshape}
	Design and optimise a Hybrid UAV that meets requirements and constraints by using project management and systems engineering tools, by a team of 10 students within 11 weeks.
	\end{itshape}
\end{quote}

The required functions of the Hybrid UAV identified and defined in the baseline report form the foundation upon which the requirements and constraints are determined \cite{baseline}. These functions and requirements follow indirectly from the mission need statement (MNS) which concisely defines what the goal of the design is and is characterised as follows:

\begin{quote}
\begin{itshape}
Carry out both remotely controlled and supervised autonomous monitoring and transport missions, comprising vertical take-off and landing, and sustained high-velocity horizontal flight.
\end{itshape}
\end{quote}

\autoref{ch:proj_desc} defines the required functions of the UAV and as a result gives guidance as to what outcome is required of the design process. \autoref{ch:proj_appr} defines the approach to be taken to get to the desired result by outlining what preparation, tools, and verification \& validation procedures are necessary. \autoref{ch:mark_stak_anal} analyses both the market and the direct stakeholders, and \autoref{ch:tech_requ} defines the general system requirements. \autoref{ch:conc_trad} briefly discusses the basis of the baseline report (such as the trade-off) and reveals the winning concept, and \autoref{ch:syst_desc} defines the subsystems of the chosen concept and their interrelations. Chapters \ref{ch:powe_prop} to \ref{ch:stru_anal} present the Power \& Propulsion, Aerodynamic, Stability \& Control, and Structural analyses respectively, and \autoref{ch:avio_grou_hand} layouts what avionics and electronic components are necessary to fulfil the requirements. \autoref{ch:desi_spec} presents the mass, cost and power breakdowns of the design, and \autoref{ch:desi_spec} analyses and outlines what procedures need to be taken during operations. \autoref{ch:prod_plan} outlines the production plan, and \autoref{ch:susdevlca} presents the life cycle assessment. Finally \autoref{ch:conc_reco} presents the conclusions of this report, and suggests recommendations for further research.


